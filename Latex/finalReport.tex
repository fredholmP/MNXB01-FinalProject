\documentclass[a4, 12pt]{article}

\usepackage[margin = 2cm]{geometry}
\usepackage{graphicx}
\usepackage{caption}
\usepackage{subcaption}
\usepackage{float}
\usepackage{hyperref}
\setcounter{secnumdepth}{0}

\usepackage{amsmath}
\usepackage{amssymb}
\usepackage{siunitx}
\usepackage{tcolorbox}
\usepackage{xcolor}
\usepackage{listings}

\definecolor{dkgreen}{rgb}{0,0.6,0}
\definecolor{gray}{rgb}{0.5,0.5,0.5}
\definecolor{mauve}{rgb}{0.58,0,0.82}
\lstset{frame=tb,
  language=c++,
  aboveskip=3mm,
  belowskip=3mm,
  showstringspaces=false,
  columns=flexible,
  basicstyle={\small\ttfamily},
  numbers=none,
  numberstyle=\tiny\color{gray},
  keywordstyle=\color{blue},
  commentstyle=\color{dkgreen},
  stringstyle=\color{mauve},
  breaklines=true,
  breakatwhitespace=true,
  tabsize=3
}



\title{Measuring Climate Trends using ROOT}
\author{MNXB01 \\ Philip J. Fredholm \\Lee Chun Hin }



\begin{document}
\maketitle
\tableofcontents
\newpage

\section{Introduction}
As discussed talked about with Oxana, we will only put our plots into this draft report and update it later with the text.


\section{Theory}
\section{Method}
\section{Results}


%Philip's Figures
\begin{figure}[H]
\centering
\includegraphics[scale=0.50]{philipCold.pdf}
\caption{Shows the number of occurrences of a specific day being the coldest day of that year in Borås. Negative days indicate that the day was in the end of the previous year.}
\end{figure}


\begin{figure}[H]
\centering
\includegraphics[scale=0.50]{philipHot.pdf}
\caption{Shows the number of occurrences of a specific day being the warmest day of that year in Borås.}
\end{figure}

\begin{figure}[H]
\centering
\includegraphics[scale=0.50]{philipSummer.pdf}
\caption{Shows the number of occurrences of a specific day being the first day of summer that year in Borås.}
\end{figure}
\newpage




%Chris's Figures
\begin{figure}[H]
\centering
\includegraphics[scale=0.6]{chrisFig1.pdf}
\caption{Temperature of Umeå Airport in 23/8 from 1962 to 2020.}
\end{figure}


\begin{figure}[H]
\centering
\includegraphics[scale=0.7]{chrisFig3.pdf}
\caption{ Temperature of Umeå Airport in each of the date of 23 since 1962 to 2020.}
\end{figure}




%Figure 1.1 shows the temperature for each month of 23/8 from 1962 to 2020.The mean temperature and the standard deviation is 16.37 and3.254 respectively.We can predict the probability of particular tmeperature by using the mathematical below:
%Z=(X-μ)/σ where Z=Standard score,,σ=standard deviation,μ=population mean
%For example if we want to predict what is the probability for the coming 23/8 to be in the temperature of 12°c.By applying the above equation,
%Z=(12-16.37)/3.254
%Z=-1.34
%By looking the table below(figure1.2), we can observe the probability of different value of standard score.

%In this case, the probabilty will be in0.09.

%Figure 1.2 show each of the date 23 in every month since 1962 to 2020. The mean and the standard deviation are 5.662 and 10.43 respectively.
%As discussed above, we can use the same formula to predcit the temperature of the coming date 23 for having specific temperature.

%Figure 1.1 having a lower standard deviation than figure 1.2, which means that the data in fig.1,1 are more less disperse and more closing to the mean, which lead to a more focusing graph.




\section{Discussion}
%In order to make the histogram, i have use of the excel and root. Firstly, I use the finding function in excel to locate the string or the data I need. However, it is in a string format, and i need to separate the temperature data from the string. Hence i make use of the data analysis function in excel. It can sepraate the string by detecting each of the symbol(such as ;,) in the string. As a result i can seprate the temperature data I need from the original excel file.After that I use Win Zip to transfer the revised data to Auora. For the coding part, I create a new canvas named as c1 and with the title of Temperature of Umeå FLyplats. SInce I want to show the two graphs in the same page and in parllel, so i make use of the function of Divide(2,1). After that I create a histogram named as hist with the x,y axis be [°c], and Entries respectively. And the pixels of the histogram is set to 700. Besdies, the range of x axis is from -30 to 40. Then, I try to input the data file which named as 8-23Temp.txt into the histogram by using the command ifstream infiles and infiles.open. After that i store the data in a variable named as tmp by using the command double.
\section{References}
[1] The lab manual \textit{LAB MANUAL TITLE}, given out by Lund University in association with the course COURSE NUMBER during the SEASON semester of YEAR. \newline \newline
\noindent 






\end{document}
